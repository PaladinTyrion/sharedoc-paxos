\section{Introduction}

Real-time online collaboration software like Etherpad and Google Docs faces the challenge of
maintaining the consistency of a concurrently updated shared state without resorting to the
old-school lock-based technique that can result in blocking. Operational Transformation \cite{sun1998operational}
was proposed to address exactly this problem. Despite its nice properties such as lock-free and
non-blocking, implementing OT-based systems can be a real pain. An (alleged) former engineer at
Google revealed that an OT implementation at Google took two years and it can still take as long if
it were to be redone in the future \cite{sharejs}.

We have this exact feeling while we read through documentations of open source solutions like
Etherpad as well as some academic papers describing OT algorithms. They are either too high-level
and therefore too vague to be used to actually come up with a correct implementation, or too
convoluted to be easily understood, let alone intuitively reasoning about correctness. For this
reason we decide to develop our own simple, but fully functional, OT algorithm to develop intuitions
about this powerful framework.

During our search of a more understandable OT algorithm, we made the observation that most existing
algorithms focused too much on the ``merge'' routine of the algorithm. In an OT algorithm, ``merge''
routine must ensure that different input orders of the same set of operations results in the same
output. This is critical to ensure that an eventual consistent view is presented to all clients
editing the same document, because for different clients, changes made by others can arrive in
different order. Designing a mathematical solution to transform normally non-commutative document
edits into commutative operations that are safe to merge in different orders is a challenging task,
and also makes many OT algorithms obscure and difficult to understand.

In order to develop intuitions about an OT algorithm, we chose to follow a different path: focus on
the ``disambiguation'' part of the OT framework. We think that as long as we can precisely capture
the intention of each edit operation and reconcile them in a globally consistent order, it should
not be difficult to come up with an intuitive solution. We applied this idea throughout the design
of our system and observed positive result in implementation and testing. We will describe the
detailed design of our OT-based online collaboration system as well as the intuition we gained from
building it in the rest of this document.
