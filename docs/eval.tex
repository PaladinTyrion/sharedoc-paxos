\section{Testings and Lessons Learned}

We evaluated the correctness of the system by manual testing. To test the
client-side program, we have also written a JavaScript version of a unreplicated
single-threaded server using the Node.JS framework~\cite{nodejs}. Once we are
convinced that the client-side script is working, we integrate it with our
Paxos-enabled servers. So far we have not encountered any errors in our testing
with multiple clients connected to different Paxos peers. We also did not
observe any perceptible delays of clients seeing each other's updates in those
tests.

We also found our debugging process (from first compile to being fully
functional) relatively smooth. Most of the problems identified and fixed during
this process were related to client-server communication rather than the core
program logic. Communication caused us a little trouble because JavaScript has a
dynamic type system while Go has a static one. We worked this problem around by
forcing everything to string before converting a JavaScript object to JSON. Our
relatively effortless debugging process is well reflected in the commit history
in the project repository.

The major lesson we learned by doing this project is that it always helps to
think carefully about the mathematical properties of a design before actually
``hack'' into it. We spent a lot of time discussing possible solutions and
abandoned many approaches that first seemed trivial but turned out to require
ad-hoc reasoning for many corner cases. In the end we choose the design that has
the strongest correctness properties: let both clients and servers agree on a
global commit order, rather than letting them diverge first and reconcile
committed versions later. Our relatively effortless debugging process highlights
the importance of thinking hard about the right design.

Another lesson we learned is that there are two problems an OT algorithm needs
to address: to precisely interpret the intentions of the operations especially
when they arrive out-of-order, and to reasonably merge all operations. It helps
to keep this fact in mind, and any working algorithm must address both problems.
Focusing on any single problem isn't going to help much and is likely to make
the resulting solution obscure and with lots of corner cases to address. In
hindsight, it makes sense to only merge intentions together in an on-line
collaboration system, rather than finding a clever way to merge ``diffs''
together.
