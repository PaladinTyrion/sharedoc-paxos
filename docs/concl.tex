\section{Future Work and Conclusion}

We acknowledge that there are still significant limitations to our current
implementation (as of project submission). We don't handle persistence in our
system so everything is lost when the server is shut down, and clients currently
can't work on separate documents\footnote{Servers do maintain separate states
for different documents so this feature is actually implemented on the server
side. To simplify our user interface so that we can focus on the distributed
system part of the system, we decided not to add this feature to the client-side
script.}. Another limitation comes from our way of generating operations: our
current approach of listening to key strokes doesn't work if the client uses
copy-and-paste or deletes a block of selected text. These are not fundamental
design issues but they do represent some usability issues. We plan to address
some of these problems in the future to make it more suitable for real-world
use.

In summary, we implemented a distributed version of collaborative editing
platform with Paxos-backed servers. We also designed our own operational
transformation (OT) algorithm used to keep all clients and servers in sync,
which shows strong correctness properties by allowing uncommitted operations to
mutate and ensuring a unique order for all committed operations. In our system,
different clients can connect to any server in the Paxos quorum and collaborate
on a shared document without perceptible delays. Our development experience
demonstrates the importance of a good design, and our testing results show that
our design is functionally correct and has decent performance.
