%-----------------------------------------------------------------------------
%
%               Template for sigplanconf LaTeX Class
%
% Name:         sigplanconf-template.tex
%
% Purpose:      A template for sigplanconf.cls, which is a LaTeX 2e class
%               file for SIGPLAN conference proceedings.
%
% Guide:        Refer to "Author's Guide to the ACM SIGPLAN Class,"
%               sigplanconf-guide.pdf
%
% Author:       Paul C. Anagnostopoulos
%               Windfall Software
%               978 371-2316
%               paul@windfall.com
%
% Created:      15 February 2005
%
%-----------------------------------------------------------------------------


\documentclass[nocopyrightspace]{sigplanconf}

% The following \documentclass options may be useful:

% preprint      Remove this option only once the paper is in final form.
% 10pt          To set in 10-point type instead of 9-point.
% 11pt          To set in 11-point type instead of 9-point.
% authoryear    To obtain author/year citation style instead of numeric.

\usepackage{amsmath}
\usepackage{graphicx}
\usepackage{url}

\begin{document}


\special{papersize=8.5in,11in}
\setlength{\pdfpageheight}{\paperheight}
\setlength{\pdfpagewidth}{\paperwidth}

%\conferenceinfo{CONF 'yy}{Month d--d, 20yy, City, ST, Country} 
%\copyrightyear{20yy} 
%\copyrightdata{978-1-nnnn-nnnn-n/yy/mm} 
%\doi{nnnnnnn.nnnnnnn}

% Uncomment one of the following two, if you are not going for the 
% traditional copyright transfer agreement.

%\exclusivelicense                % ACM gets exclusive license to publish, 
                                  % you retain copyright

%\permissiontopublish             % ACM gets nonexclusive license to publish
                                  % (paid open-access papers, 
                                  % short abstracts)

% \titlebanner{banner above paper title}        % These are ignored unless
% \preprintfooter{short description of paper}   % 'preprint' option specified.

\title{Experimenting with Operational Transformations in an Etherpad-like Online Collaboration System}

\authorinfo{Yihe Huang}
  {Harvard University}
  {hyihe@mit.edu\\yihehuang@g.harvard.edu}
\authorinfo{Yongbin Sun}
  {Massachusetts Institute of Technology}
  {yb\_sun@mit.edu}
% \subtitle{Subtitle Text, if any}

\maketitle

\begin{abstract}

For an online collaboration platform like Etherpad, simply resolving conflicts
on the server side is not enough. The underlying algorithms should also be
responsible for presenting eventual consistent views to all collaborators,
while allowing them to briefly diverge in order to avoid blocking. We built
such a system based on the idea of operational transformation to understand
the challenges of designing OT-based systems. Our results show that simple OT
designs have very strong and intuitive correctness properties that beat most
ad-hoc reasonings frequently seen in distributed systems.

\end{abstract}

% \category{CR-number}{subcategory}{third-level}

% general terms are not compulsory anymore, 
% you may leave them out

\keywords
Operational transformation, conflict resolution, distributed systems

\section{Introduction}

Real-time online collaboration software like Etherpad~\cite{etherpad} and Google Docs~\cite{gdocs}
faces the challenge of maintaining the consistency of a concurrently updated shared state without
resorting to the old-school lock-based technique that can result in blocking. Operational
Transformation \cite{sun1998operational} was proposed to address exactly this problem. Despite its
nice properties such as lock-free and non-blocking, implementing OT-based systems can be a real
pain. An (alleged) former engineer at Google revealed that an OT implementation at Google took two
years and it can still take as long if it were to be redone in the future \cite{sharejs}.

We have this exact feeling while we read through documentations of open source solutions like
Etherpad as well as some academic papers describing OT algorithms. They are either too high-level
and therefore too vague to be used to actually come up with a correct implementation, or too
convoluted to be easily understood, let alone intuitively reasoning about correctness. For this
reason we decide to develop our own simple, but fully functional, OT algorithm to develop intuitions
about this powerful framework.

During our search of a more understandable OT algorithm, we made the observation that most existing
algorithms focused too much on the ``merge'' routine of the algorithm. In an OT algorithm, ``merge''
routine must ensure that different input orders of the same set of operations results in the same
output. This is critical to ensure that an eventual consistent view is presented to all clients
editing the same document, because for different clients, changes made by others can arrive in
different order. Designing a mathematical solution to transform normally non-commutative document
edits into commutative operations that are safe to merge in different orders is a challenging task,
and also makes many OT algorithms obscure and difficult to understand.

In order to develop intuitions about an OT algorithm, we chose to follow a different path: focus on
the ``disambiguation'' part of the OT framework. We think that as long as we can precisely capture
the intention of each of the edit operations, and reconcile them in a globally consistent order, it
should not be difficult to come up with an intuitive solution. We applied this idea throughout the
design of our system and observed positive result in implementation and testing. We will describe
the detailed design of our OT-based online collaboration system as well as the intuition we gained
from building it in the rest of this document.

\section{Design}
\section{Implementation}

We implemented our system as a web application. The client-side program is
completely written in HTML/JavaScript and is compatible with most modern web
browsers, and the server is written in the Go programming language (most notably
the Go {\tt http} package~\cite{gohttp} to implement most of the front-end
server functionality). We also used {\tt socket.io}~\cite{socketio}, a full-
duplex client-server communication framework using {\tt websocket}
technology~\cite{websock}, instead of standalone HTTP requests/responses for a
more streamlined design. Open-source {\tt socket.io} libraries for
JavaScript~\cite{jssocketio} and Go~\cite{gosocketio} are the only external
libraries we used to implement our OT framework. We also used this code snippet
from Internet~\cite{linenumber} to improve the look of our user interface. All
source code of the project is publicly available at
\url{https://github.com/huangyihe/etherpad-paxos}.

\subsection{Client-side Script}

A JavaScript program implements all logic and functionality described in
Section~\ref{sec:design_client}. The program generates client operations by
listening on key strokes. Generated operations (which are JavaScript objects)
are converted to JSON strings before being transmitted over the network via {\tt
socket.io}.

\subsection{Server-side Paxos Integration}

To improve availability of the server, we replicate the server using the Paxos
protocol~\cite{lamport1998part}. Clients can connect to any server in the Paxos
quorum without noticing the difference. One challenge to address is to make sure
that clients connected to different Paxos peers can see each other's updates in
a timely manner. We achieve this by running a background thread at each Paxos
server (called the ``auto-apply'' thread) whose sole job is to periodically look
for new decided entries in the Paxos log and apply them if any.

\section{Testings and Lessons Learned}

We evaluated the correctness of the system by manual testing. To test the
client-side program, we have also written a JavaScript version of a unreplicated
single-threaded server using the Node.JS framework~\cite{nodejs}. Once we are
convinced that the client-side script is working, we integrate it with our
Paxos-enabled servers. So far we have not encountered any errors in our testing
with multiple clients connected to different Paxos peers. We also did not
observe any perceptible delays of clients seeing each other's updates in those
tests.

We also found our debugging process (from first compile to being fully
functional) relatively smooth. Most of the problems identified and fixed during
this process were related to client-server communication rather than the core
program logic. Communication caused us a little trouble because JavaScript has a
dynamic type system while Go has a static one. We worked this problem around by
forcing everything to string before converting a JavaScript object to JSON. Our
relatively effortless debugging process is well reflected in the commit history
in the project repository.

The major lesson we learned by doing this project is that it always helps to
think carefully about the mathematical properties of a design before actually
``hack'' into it. We spent a lot of time discussing possible solutions and
abandoned many approaches that first seemed trivial but turned out to require
ad-hoc reasoning for many corner cases. In the end we choose the design that has
the strongest correctness properties: let both clients and servers agree on a
global commit order, rather than letting them diverge first and reconcile
committed versions later. Our relatively effortless debugging process highlights
the importance of thinking hard about the right design.

Another lesson we learned is that there are two problems an OT algorithm needs
to address: to precisely interpret the intentions of the operations especially
when they arrive out-of-order, and to reasonably merge all operations. It helps
to keep this fact in mind, and any working algorithm must address both problems.
Focusing on any single problem isn't going to help much and is likely to make
the resulting solution obscure and with lots of corner cases to address. In
hindsight, it makes sense to only merge intentions together in an on-line
collaboration system, rather than finding a clever way to merge ``diffs''
together.

\section{Future Work and Conclusion}

We acknowledge that there are still significant limitations to our current
implementation (as of project submission). We don't handle persistence in our
system so everything is lost when the server is shut down, and clients currently
can't work on separate documents\footnote{Servers do maintain separate states
for different documents so this feature is actually implemented on the server
side. To simplify our user interface so that we can focus on the distributed
system part of the system, we decided not to add this feature to the client-side
script.}. Another limitation comes from our way of generating operations: our
current approach of listening to key strokes doesn't work if the client uses
copy-and-paste or deletes a block of selected text. These are not fundamental
design issues but they do represent some usability issues. We plan to address
some of these problems in the future to make it more suitable for real-world
use.

In summary, we implemented a distributed version of collaborative editing
platform with Paxos-backed servers. We also designed our own operational
transformation (OT) algorithm used to keep all clients and servers in sync,
which shows strong correctness properties by allowing uncommitted operations to
mutate and ensuring a unique order for all committed operations. In our system,
different clients can connect to any server in the Paxos quorum and collaborate
on a shared document without perceptible delays. Our development experience
demonstrates the importance of a good design, and our testing results show that
our design is functionally correct and has decent performance.


% We recommend abbrvnat bibliography style.

\bibliographystyle{abbrvnat}

\bibliography{main}

% The bibliography should be embedded for final submission.

\end{document}

%                       Revision History
%                       -------- -------
%  Date         Person  Ver.    Change
%  ----         ------  ----    ------

